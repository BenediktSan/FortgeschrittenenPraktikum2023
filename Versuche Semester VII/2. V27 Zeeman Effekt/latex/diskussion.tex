\newpage
\section{Diskussion}

\noindent In dem Versuch wurde zuerst das Magnetfeld kalibriert. Es wird eigentlich von einem linearen Zusammenhang ausgegangen, jedoch ist ein 
Ploynom-Fit der kubischen Ordnung deutlich passender. Die Abflachung der Steigung erklärt sich durch die Erwärmung des Magneten. Es wurden eine 
Hystere-Messung durchgeführt, die Werte bei der absteigenden Messung liegen immer etwas höher als ihre entsprechenden Werte. Dies kann mit einer 
Restmagnetisierung und Erwärmung der Polschuhe erklärt werden. \\

\noindent 
Anschließend hat die Einstellung der Linsen und Blende eine sehr lange Zeit gedauert und erst nach einigem Probieren ein zufriedenstellendes Ergebnis 
geliefert. Das Machen der Fotos hat keine größeren Probleme aufgezeigt, jedoch musste aufgepasst werden, die Kamera beim Drücken des Auslösers nicht 
zu stark zu bewegen. Dies ist wichtig um die $\Delta s_i$ den jeweiligen $\delta s_i$ zuordnen zu können. \\

\noindent
Die Berechnung der Land\'{e}-Faktoren liefert zufriedenstellende Ergebnisse, diese sind in \autoref{tab:rel} aufgelistet. Es ist zu sehen, dass 
der berechnete Land\'{e}-Faktor für den $\sigma$ Übergang der blauen Spektrallinie am meisten von seinem Theoriewert abweicht. Dies liegt 
daran, dass es sich hier um eine Überlagerung aus der $g = \num{2}$ und $g = \num{1.5}$ Aufspaltungen handelt. Somit ist das Ablesen der klaren 
Grenzen der Linie in den Fotos schwierig. 


\begin{table}[ht]
    \centering
    \caption{Relative Abweichung von den Theoriewerten für die einzelnen Berechnungen}
    \label{tab:rel}
        \begin{tabular}{c S[table-format=1.2]@{${}\pm{}$}S[table-format=1.2] S[table-format=1.1] S[table-format=2.2]}
        \toprule
        & \multicolumn{2}{c}{$g_\text{exp}$} & {$g_\text{theo}$} & {$\increment g \mathbin{/} \si{\percent} $} \\
        \midrule
        rot, $\sigma$ & 1.02 & 0.13 & 1 & 2.02\\
        blau, $\sigma$ &  1.93 & 0.13 & 1.75 & 10.25 \\
        blau, $\pi$ & 0.52 & 0.08 & 0.5 & 4.75 \\
        \bottomrule
    \end{tabular}
\end{table} 

