\newpage
\section{Diskussion}

\noindent
Die Durchführung des Versuchs lief ohne Probleme ab. Es konnte für die Operationsverstärker symmetrische Betriebsspannungen eingestellt werden.\\
Für die Auswertung des Linearverstärkers gab es keine Probleme, bis auf die Grenzfrequenzen, die teilweise Abweichungen besitzen, die nominal größer sind als die errechneten Frequenzen.
Dies lässt sich vermutlich darauf zurückführen, dass bei der Berechnung logarithmiert, Exponentiert und dividiert wird, wodurch die Fehlerfortpflanzung einen großen Einfluss besitzt.
Die errechneten Leerlaufverstärkungen besitzen aber, wie in Tabelle \ref{tab:rel} zu erkennen, nur eine geringe Abweichung von den Theoriewerten.\\
Der für den Integrator errechnete Wert besitzt zwar eine große relative Abweichung, bewegt sich aber in der richtigen Größenordnung. 
Die Grafiken der integrierten Funktionen entsprechen den Erwartungen.\\
Für den Differenzierer sind die Abweichungen von der Theorie deutlich kleiner, allerdings entsprechen die Grafiken für die Dreiecksspannung und die Rechteckspannung nicht den Erwartungen.
Dies ist in den Abbildungen \ref{fig:diff_dre} und \ref{fig:diff_recht} zu finden. 
Da das Ergebnis für die Sinusspannung stimmt, ist nicht davon auszugehen, dass etwas falsch verkabelt ist. 
Dies wird auch dadurch unterstrichen, dass die Werte nah an der Theorie liegen und zur vorigen Messung nur zwei Bauteile ausgetauscht werden mussten.
Ein Grund könnten langsame Entladungen des Kondensators sein, wodurch keine klaren Peaks entstehen, sondern die auftretenden Entladungen. 


\begin{table}[H]
    \centering
    \caption{Relative Abweichung von den errechneten Theoriewerten für die einzelnen errechneten Parameter.
    Dabei werden die Ergebnisse in der Reihenfolge invertierter Linearverstärker, Integrator, Differenzierer, Schmitt-Trigger und Signalgenerator aufgeführt. }
    \label{tab:rel}
        \begin{tabular}{S [table-format=4.4] S [table-format=4.4] S [table-format=4.4] S [table-format=4.4]}
        \toprule
            {Größe} & {Parameter} &{Literaturwert}& {$\text{relative Abweichung} $} \\
        \midrule
        ${V_1}$   & $\SI{10.27(13)}{}$                     & $\SI{10}{}$                & $\SI{-2.70}{\percent}$\\      
        ${V_2}$   & $\SI{96.00(115)}{}$                     & $\SI{100}{}$                & $\SI{4}{\percent}$\\
        ${V_3}$   & $\SI{67.45(140)}{}$                     & $\SI{68}{}$                & $\SI{0.81}{\percent}$\\\\
        \hline
        ${RC_\t{Int}}$   & $\SI{2.9(3)}{\milli\second}$  & $\SI{1}{\milli\second}$    & $\SI{-192.76 (3200)}{\percent}$\\\\
        \hline
        ${RC_\t{Diff}}$   & $\SI{2.4(3)}{\milli\second}$ & $\SI{2.2}{\milli\second}$    & $\SI{ -10.76(1504)}{\percent}$\\\\
        \hline
        $U_\t{Kipp,+}$   & $\SI{1.86}{\volt}$      & $\SI{1.4}{\volt}$    & $\SI{  -33.04}{\percent}$\\
        $U_\t{Kipp,-}$   & $\SI{-1.669(6)}{\volt}$ & $\SI{1.4}{\volt}$    & $\SI{ -19.20(045)}{\percent}$\\\\
        \hline
        $U_\t{max}$       & $\SI{2.65}{\volt}$        & $\SI{1.375}{\volt}$         & $\SI{  80.73}{\percent}$\\
        $f_\t{Dreieck}$   & $\SI{1.64}{\kilo\hertz}$ & $\SI{2.5}{\kilo\hertz}$    & $\SI{ 34}{\percent}$\\\\
        \bottomrule
    \end{tabular}
    \label{tab:rel}
\end{table} 

\noindent
Die Ergebnisse des Schmitt-Triggers liegen über den errechneten, sind aber trotzdem in derselben Größenordnung.
Dieses Ergebnis könnte verbessert werden, wenn für jeden der Werte über mehr als zwei Messwerte gemittelt werden würde.\\
Der Signalgenerator liefert wie zu erwarten ein Dreieckssignal. Allerdings ist die gemessene Frequenz größer und die maximale Amplitude kleiner als die 
theoretisch berechneten. Dies könnte sich darüber erklären lassen, dass hier ein sehr komplizierter Aufbau betrachtet wird, 
in dem nicht betrachtete Innenwiderstände, Leerlaufspannungen oder Toleranzbereiche, einen größeren Einfluss haben können.\\
Alles in allem liegen die Ergebnisse aber nah an der Theorie und sind somit zufriedenstellend.
