\newpage
\section{Aufbau und Durchführung}

    \noindent 
    Zur Durchführung des Versuches stehen ein Frequenzgenerator, ein Spannungsgenerator und ein elektronisches Oszilloskop zur Verfügung. 
    Die Schaltungen werden mithilfe von entsprechenden Steckelementen auf einer Steckplatte realisiert. \\ 
    Alle Messungen wurden bei einer Betriebsspannung von $U_\text{S} = \SI{15}{\volt}$ gemacht.

    \subsection{Invertierter Linearverstärker}

        \noindent 
        Es wird die Schaltung, wie sie in \autoref{fig:invertierter_lin} zu sehen ist, aufgebaut. Dabei werden die Widerstände $R_1 = \SI{1}{\kilo\ohm}$ und 
        $R_2 = \SI{100}{\kilo\ohm}$ verwendet. Dann wird die Frequenz der Eingangsspannung erhöht, sodass über mehrere Dekaden Werte aufgenommen werden, und bei 
        den verschiedenen Frequenzen jeweils die Amplitude der Ausgangsspannung und die Phasendifferenz zwischen Ein- und Ausgangsspannung erfasst. \\
        Das gleiche Vorgehen wird für zwei andere Widerstandsverhältnisse wiederholt. 

    \subsection{Umkehrintegrator}

        \noindent 
        Die Schaltung wird analog zu \autoref{fig:umkehrint} aufgebaut mit $R = \SI{10}{\kilo\ohm}$ und $C = \SI{100}{\nano\farad}$. 
        Es werden für verschiedene Frequenzen bei einer Sinuseingangsspannung die Amplitude der Ein- und Ausgangsspannungen gemessen. 
        Anschließend werden für eine Sinus- , eine Dreiecks- und eine Rechteckspannung Fotos am Oszilloskop gemacht und die Rohdaten gespeichert.

    \subsection{Invertierter Differenzierer}

        \noindent
        Die Schaltung wird nach \autoref{fig:inverdiff} aufgebaut, wobei der Widerstand hier $R= \SI{100}{\kilo\ohm}$ beträgt und der Kondensator 
        $C = \SI{22}{\nano\farad}$. Analog zum Vorgehen beim Umkehrintegrator werden bei einer sinusförmigen Eingangsspannung für verschiedene 
        Frequenzen die Amplituden der Ein- und Ausgangsspannung aufgenommen. Auch hier werden jeweils für eine Sinus-, Dreieck- und Rechteckspannung 
        Fotos mit dem Oszilloskop gemacht und die Rohdaten gespeichert.

    \subsection{Nichtinvertierter Schmitt-Trigger}

        \noindent 
        Es wird die Schaltung analog zur \autoref{fig:schmitttrig} mit den Widerständen $R_1 = \SI{10}{\kilo\ohm}$ und $R_2 = \SI{100}{\kilo\ohm}$ 
        aufgebaut. 
        Es wird nun die Amplitude der Eingangsspannung langsam erhöht, bis die Ausgangsspannung kippt. Dieser Schwellenwert wird notiert. 
        Sonst kann auch eine Dreiecksspannung mit einer Amplitude größer als der vermutete Schwellenwert eingestellt werden und dann die 
        Schwellspannung mit dem Oszilloskop an mehreren Stellen ausgemessen werden. 

    \subsection{Signalgenerator}

        \noindent 
        Die Schaltung wird anhand \autoref{fig:signal} aufgebaut mit den Bauteilen $R_1 = \SI{10}{\kilo\ohm}$, $R_2 = \SI{100}{\kilo\ohm}$, 
        $R_3 = \SI{1}{\kilo\ohm}$ und $C = \SI{1}{\micro\farad}$. 
        Es werden am Oszilloskop die Spannungen $U_1$ und $U_\text{A}$ beobachtet und als Bild und Rohdatei gespeichert. 