\newpage
\section{Diskussion}

\noindent
Eine große Fehlerquelle des Versuchs ist die Temperatur des Magneten, die sich natürlich auch auf die der Probe auswirkt. 
Da die Magnetfeldstärke stark vom Stromfluss abhängig ist und dieser von der Temperatur abhängt, ergibt sich ein großer Faktor, der in der Auswertung nicht betrachtet wurde. 
Allerdings trat während der Messungen keine Änderung in der Larmor-Frequenz auf, welche sehr stark vom Magnetfeld des Permanentmagneten und damit der Temperatur abhängig ist.
Deswegen kann davon ausgegangen werde, dass die Temperatur des Permanentmagneten über die Messungen konstant ist.
%Die Temperatur des Magneten wirkt sich außerdem auf die Temperatur der Probe aus, was zu Schwankungen in Dichte, Viskosität und Diffusionskonstante führen kann. 
Die gemessene Temperatur zwischen den Magnetpolen war zu allen Zeitpunkten $T= \SI{20.8}{\degreeCelsius}$. 
Dies ist erwartet, da die Temperatur mit einem Peltierelement stabilisiert wird. 
Deswegen wird angenommen, dass Konstanten wie Dichte, Viskosität oder die Diffusionskonstante für alle Messungen konstant sind.\\
Ein weiterer Faktor der zu Abweichungen führen kann, sind Verunreinigungen außen an der Probe, wie Fett von Fingern, oder andere Elemente in der Probe. 
Dazu würden zum Beispiel gelöste Gase in Wasser gehören, welche die Messungen beeinflussen können. \\\\

\begin{table}[H]
    \centering
    \caption{Relative Abweichung von den Literaturwerten \protect\cite{theo}\protect\cite{radius} für die einzelnen über die Halbwertsbreite errechneten Parameter.
    Die Relaxationszeiten wurden dabei bei Raumtemperatur gemessen.}
    \label{tab:rel}
        \begin{tabular}{S [table-format=4.4] S [table-format=4.4] S [table-format=4.4] S [table-format=4.4]}
        \toprule
            {Größe} & {Fit-Parameter} &{Literaturwert}& {$\text{relative Abweichung} $} \\
        \midrule
        ${T_1}$   & $\SI{2.84(7)}{\second}$                     & $\SI{3.09(15)}{\second}$                & $\SI{8.09(497)}{\percent}$\\
        ${T_2}$   & $\SI{1.98 (10)}{\second}$                    & $\SI{1.52(9)}{\second}$                 & $\SI{-30(10)}{\percent}$\\\\
            \hline
        ${D_\t{FWHM}}$   & $\SI{ 1.30(1)e-9}{\metre^2\per\second}$     & $ \SI{2.78(4)e-9}{\metre^2\per\second}$   & $\SI{53.40(77)}{\percent}$\\\\
            \hline
        ${r_\t{FWHM}}$     & $\SI{1.174(9)e-11}{\metre}$                & $\SI{1.69e-10}{\metre}$                  & $\SI{93.25(5)}{\percent}$\\
        \bottomrule
    \end{tabular}
    \label{tab:rel}
\end{table} 

\noindent
In der Tabelle \ref{tab:rel} sind die relativen Abweichungen der Messwerte von den Literaturangaben aus den Quellen \cite{theo} und \cite{radius} dargestellt.
Dabei ist zu erkennen, dass sich alle Werte, bis auf den Radius, in den korrekten Größenordnungen bewegen. 
Für die Relaxationszeiten wurden die Literaturangaben bei Raumtemperatur gemessen. Dabei wurde keine Angabe gemacht ob die Proben entgast waren.
Die Temperatur deckt sich dadurch vermutlich mit der unserer Probe, allerdings wurde keine Angabe gemacht ob das Wasser auch bidestilliert ist.\\
Für die Relaxationszeit $T_1$ ist die relative Abweichung sehr gering. 
Dies zeigt, das trotz nicht beachteter Verunreinigungen unserer Probe genaue Ergebnisse möglich sind. 
Die Genauigkeit ist außerdem überraschend, da die Messreihe nicht bis weit nach denn Nulldurchgang der Spannungsamplitude fortgesetzt werden konnte. 
Dies ist in Abbildung \ref{img:T1} zu erkennen.\\
Für $T_2$ zeigt sich eine größere Abweichung von der Literatur. Diese ist aber in Anbetracht der nicht betrachteten Faktoren verhältnismäßig klein.
Eine Erklärung für die Abweichung könnte die Funktion \textit{scipy.find\_peaks} sein, 
die eventuell nicht den genauen Ort der Peaks gefunden hat. Dies ist allerdings unwahrscheinlich.\\
Bei der Messung des Echo-Signals wurden die Messdaten nicht bis zum Verschwinden der letzten Schwingungen aufgenommen.
Daher ist die Auflösung im Frequenzraum zu gering um über die Breite eine ausreichend genaue Bestimmung des Magnetfeldgradienten $G$ durchzuführen.
Die deshalb zusätzlich durchgeführte Bestimmung über die Halbwertsbreite des Signals 
führt dabei allerdings trotzdem auf einen ähnlichen Wert wie die Bestimmung über das fouriertransformierte Echo.
Dabei ist zu beachten, dass die Halbwertsbreite nur grafisch bestimmt wurde, ebenso wie die Breite der Röhre im Frequenzraum $d_\t{f}$, 
welche nur ungenau abgeschätzt werden konnte.
Dies ist ein Faktor der zu der großen Abweichung beitragen kann.\\
Bei der Bestimmung der Diffusionskonstante $D$ wurde zuerst Überprüft, ob die Messwerte einer $\tau^3$ Abhängigkeit folgen.
Dabei wurde festgestellt, dass diese nur auf Intervallen vorliegt, global allerdings nicht.
Dies ist in Abbildung \ref{img:diff1} zu finden. 
Die große Abweichung des Werts lässt sich unter anderem darüber erklären. 
Außerdem wurde die Diffusionskonstante über die fehlerbehafteten Größen $T_2$ und $D$ bestimmt, was zu einer zusätzlichen Abweichung führt. 
Der Molekülradius ist wiederum von $D$ abhängig, wodurch sich ein Teil der Abweichung erklären lässt. 
Die Abweichungen von dem Literaturangaben und der dichtesten Kugelpackung kann zu großen Teilen über Fehlerfortpflanzung begründet werden.


\noindent
Trotz der vielen möglichen Faktoren die Abweichungen in den Messergebnissen hervorrufen können, liefert der Versuch gute Ergebnisse, was vorallem an $T_1$ zu sehen ist.
