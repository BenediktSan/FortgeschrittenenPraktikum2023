\section{Zielsetzung}

    \noindent In diesem Versuch sollen mit Hilfe des Verfahrens von gepulster Kernspinresonanz (NMR) die Relaxationszeiten und die Diffusionskonstante einer Probe gemessen werden. 
    Dabei wird ein besonderer Wert auf die experimentelle Durchführung und auf die Pulsmethodik gelegt.  


\section{Theorie}
\label{sec:theorie}

    \noindent Die Probe wird in ein möglichst homogenes Magnetfeld $\vec{B}_0$ gelegt, sodass sich nach dem Zeeman-Effekt die Energieniveaus aufspalten und sich eine endliche Magnetisierung 
    \begin{equation*}
        M = \sum_i \vec{\mu}_i
    \end{equation*}
    ergibt, welche sich aus den einzelnen magnetischen Dipolmomenten $\vec{\mu}_i$ zusammenaddiert. Das Kernspinsystem befindet sich in der Ruhelage im thermodynamischen Gleichgewicht, die relativen Besetzungszahlen 
    der Energiezustände ist proportional zum Boltzmann-Faktor.

    \subsection{Das Vektorbild}

        \noindent Da ein Zwei-Niveau-System betrachtet wird, kann das klassische Vektorbild genutzt werden. Es ergibt sich, ähnlich zu einer Kreiselbewegung, die Gleichung 
        \begin{equation*}
            \frac{\symup{d} \vec{\mu}}{\symup{d} t} = \gamma \vec{\mu} \times \vec{B}_0\, ,
        \end{equation*}
        wobei hier $\gamma$ für das gyromagnetische Verhältnis steht. \\ 

        \noindent Da im Laborsystem das magnetische Dipolmoment um das in die $z$-Achse zeigende Magnetfeld präzediert, ist es sinnvoll eine Koordinatentransformation in ein rotierendes Koordinatensystem (RKS) 
        der Kreisfrequenz $\vec{\omega}_\text{RKS}$ durchzuführen. Mit $\increment \vec{\omega} = \vec{\omega}_0 - \vec{\omega}_\text{RKS}$ gilt dann 
        \begin{equation*}
            \left(\frac{\symup{d}}{\symup{d}t} \vec{\mu}\right)_\text{RKS} = \increment \vec{\omega} \times \vec{\mu}\, .
        \end{equation*} 
        Es ist leicht zu sehen, dass im Resonanzfall $\vec{\omega}_0 = \vec{\omega}_\text{RKS}$ das magnetische Dipolmoment im RKS stationär ist. \\

        \noindent Ein neues magnetisches Wechselfeld $\vec{B}_1$, welches senkrecht auf $\vec{B}_0$ steht mit einer Frequenz $\omega_\text{RW}\approx \omega_0$, kann Übergänge zwischen Energieniveaus anregen. 
        Mit dem Gesamtmagnetfeld 
        \begin{equation*}
            \vec{B}(t) = \left( B_1 \sin(\omega_\text{RW} t), \, B_1 \cos(\omega_\text{RW}t), \, B_0 \right)^\top
        \end{equation*}
        kann die Präzessionsgleichung als 
        \begin{equation*}
            \left(\frac{\symup{d}}{\symup{d}t} \vec{\mu}\right)_\text{RKS} =\left(- \gamma \vec{B} \times \vec{\mu}\right)_\text{RKS}
        \end{equation*}
        geschrieben werden. Für die messbare Magnetisierung ergibt sich in Komponentenschreibweise schließlich: 
        \begin{align*}
            \left(\dot{M}_x\right)_{\text{RKS}} & = \left(- \increment \omega M_y + \omega_1M_z \right)_{\text{RKS}} \\
            \left(\dot{M}_y\right)_{\text{RKS}} & = \left(\increment \omega M_x \right)_{\text{RKS}} \\
            \left(\dot{M}_z\right)_{\text{RKS}} & = \left( - \omega_1 M_x \right)_{\text{RKS}} 
        \end{align*} 

    \subsection{Radiowellenpuls}

        \noindent Mit Hilfe des RKS ($\vec{\omega}_0 = \vec{\omega}_\text{RKS}$) kann die Zeitentwicklung der Magnetisierung nach einem Hochfrequenzpulses leicht berechnet werden. 
        Sie präzediert um das $\vec{B}_1$-Feld mit $\vec{\omega}_1 = - \gamma \vec{B}_1$. 
        Wird das Magnetfeld nun für eine Zeit $t_p$ angeschalten, so wird von einem Puls gesprochen. Für den \enquote{Drehwinkel} der Magnetisierung ergibt sich 
        \begin{equation*}
            \alpha = |\omega_1| t_p = \gamma B_1 t_p\, .
        \end{equation*}
        Mit einem $\SI{90}{\degree}_x$-Puls entspricht einem Puls, wo die Magnetisierung um die $x_\text{RKS}$ gedreht wird um den Drehwinkel $\SI{90}{\degree}$. 


    \subsection{Relaxationseffekte}

        \noindent Die longitudinale Relaxation, auch \textbf{Spin-Gitter-Relaxation} genannt, wird über $T_1$ charakterisiert. Aufgrund des anliegenden statischen Magnetfeldes gibt es eine 
        endliche Magnetisierung. Die einzelnen magnetische Momente richten sich entlang des thermodynamischen Gleichgewichtes aus. Bei dieser Ausrichtung wird Energie an das System abgegeben. \\

        \noindent Die transversale Relaxation, auch \textbf{Spin-Spin-Relaxation} genannt, wird duch $T_2$ charakterisiert. Ein $\SI{90}{\degree}$-Puls kippt die Magnetisierung in die $x$-$y$-Ebene. 
        Durch die Spin-Gitter-Relaxation verlässt die Magnetisierung diese Ebene wieder, dabei gehen Phasenbeziehungen einzelner Spins verloren. Die Quermagnetisierung muss zerfallen, da es für $t \to \infty$ 
        keine Magnetisierung senkrecht zu $\vec{B}_0$ gibt. Dieser Zerfall ist die Spin-Spin-Relaxation. Da sich die Gesamtenergie nicht ändert, aber Kohärenz verloren geht, wird auch von einem Entropieprozess 
        gesprochen. 

    \subsection{Die Blochgleichungen}

        \noindent Die mathematische Beschreibung des Systems erfolgt über die messbare Magnetisierung und bei Beachtung der Relaxationsprozesse ergeben sich die Blochgleichungen: 
        \begin{align*}
            \left(\dot{M}_x\right)_{\text{RKS}} & = \left(- \increment \omega M_y + \omega_1M_z - \frac{1}{T_2} M_x\right)_{\text{RKS}} \\
            \left(\dot{M}_y\right)_{\text{RKS}} & = \left(\increment \omega M_x - \frac{1}{T_2} M_y\right)_{\text{RKS}} \\
            \left(\dot{M}_z\right)_{\text{RKS}} & = \left( - \omega_1 M_x - \frac{1}{T_1}(M_z - M_\infty)\right)_{\text{RKS}} 
        \end{align*} 
