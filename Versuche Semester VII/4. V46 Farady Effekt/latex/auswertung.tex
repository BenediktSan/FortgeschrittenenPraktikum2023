\newpage 
\section{Auswertung}

\subsection{Bestimmung des maximalen Magnetfeldes}

    \noindent Zuerst wird das Magnetfeld am Feld der Probe ermittelt. Dazu wird das Magnetfeld in Abhängigkeit der $z$-Komponente gemessen.
    Die Werte sind in \autoref{tab:bfeld} aufgelistet. 

    \begin{table}
        \centering
        \caption{Die aufgenommenen Werte der Messung des Magnetfeldes.}
        \label{tab:bfeld}
        \begin{tabular}{S[table-format=3.0] S[table-format=4.0] | S[table-format=3.0] S[table-format=4.0] }
            \toprule
            {$z \mathbin{/} \si{\centi\metre}$} & {$B \mathbin{/} \si{\milli\tesla}$} & {$z \mathbin{/} \si{\centi\metre}$} & {$B \mathbin{/} \si{\milli\tesla}$} \\
            \midrule
            -10     & -203 & 0     &  -422 \\
            -9      & -261 & 1     &  -422 \\
            -8      & -309 & 2     &  -422 \\
            -7      & -344 & 3     &  -420 \\
            -6      & -367 & 4     &  -416 \\
            -5      & -389 & 5     &  -410 \\
            -4      & -400 & 6     &  -401 \\
            -3      & -409 & 7     &  -389 \\
            -2      & -414 & 8     &  -372 \\
            -1      & -418 & 9     &  -350 \\
            0       & -422 & 10    &  -312 \\
            \bottomrule
        \end{tabular}
    \end{table}

    \noindent Die absoluten Messwerte sind in \autoref{fig:bfeld} grafisch dargestellt. Die maximal Magnetfeldstärke entspricht der am Ort der Proben.
    Der damit korrespondierende Wert lautet
    \begin{equation} 
        B_\text{max} = \SI{422}{\milli\tesla}\quad .
    \end{equation}

    \begin{figure}[H]
        \centering
        \includegraphics[width=0.75\textwidth]{build/plots/bfeld.pdf}
        \caption{Die absoluten Werte der Magnetfeldmessung aufgetragen gegen den $z$-Abstand.}
        \label{fig:bfeld}
    \end{figure}

\subsection{Darstellung der Messergebnisse}

    \noindent Es werden für die beiden Polungsrichtungen des Magnetfeldes jeweils ein Winkel $\theta_i$ ermittelt, wie es in der Durchführung beschrieben ist. 
    Der eigentliche Winkel, mit dem die Polarisation gedreht wird, ist somit 
    \begin{equation*}
        \theta = \frac{1}{2}\left( \theta_1 - \theta_2\right)\quad .
    \end{equation*}
    Zur Ermittlung der effektiven Masse werden die gemessenen Winkel auf die Länge der Probe normiert. 

    \noindent Für die reine $\ce{GaAs}$-Probe sind die Daten in \autoref{tab:hochrein} zu finden. Für die $N = \SI{1.2e18}{\per\centi\metre\tothe{3}}$ dotierte 
    Probe sind die Winkel in \autoref{tab:probe_12} und für die mit $N = \SI{2.8e18}{\per\centi\metre\tothe{3}}$ dotierte Probe in \autoref{tab:probe_28} zu finden. \\
    In \autoref{fig:firstplot} sind die normierten Winkel $\theta_\text{frei}$ gegen $\lambda^2$ aufgetragen. 

    \begin{figure}[H]
        \centering
        \includegraphics[width=0.8\textwidth]{build/plots/firstplot.pdf}
        \caption{Die auf die Länge normierten Winkel der verschiedenen Proben.}
        \label{fig:firstplot}
    \end{figure}

    \begin{table}
        \centering
        \caption{Die gemessenen Winkel der reinen $\ce{GaAs}$-Probe der Länge $d = \SI{5.11}{\milli\metre}$. }
        \label{tab:hochrein}
        \begin{tabular}{S[table-format=1.3] S[table-format=1.3] S[table-format=1.3] S[table-format=2.3] S[table-format=3.3]}
            \toprule
            {$\lambda \mathbin{/} \si{\micro\metre}$} & {$\theta_1$} & {$\theta_2$} & {$\theta$} & {$\frac{\theta}{d}$}\\
            \midrule
            1.060 & 2.602 & 3.020 & -0.209 & -40.901 \\
            1.290 & 4.277 & 4.559 & -0.141 & -27.552 \\
            1.450 & 2.611 & 2.848 & -0.119 & -23.197 \\
            1.720 & 2.683 & 2.850 & -0.084 & -16.366 \\
            1.960 & 2.756 & 2.884 & -0.064 & -12.524 \\
            2.156 & 2.915 & 2.816 & 0.049 & 9.620 \\
            2.340 & 3.303 & 3.226 & 0.039 & 7.543 \\
            2.510 & 3.730 & 3.655 & 0.038 & 7.343 \\
            2.560 & 3.064 & 2.975 & 0.045 & 8.710 \\
            \bottomrule            
        \end{tabular}
    \end{table}

    \begin{table}
        \centering
        \caption{Die gemessenen Winkel einer $\ce{InGaAs}$-Probe der Länge $d = \SI{1.36}{\milli\metre}$ und der Dotierung $N = \SI{1.2e18}{\per\centi\metre\tothe{3}}$. }
        \label{tab:probe_12}
        \begin{tabular}{S[table-format=1.3] S[table-format=1.3] S[table-format=1.3] S[table-format=2.3] S[table-format=3.3]}
            \toprule
            {$\lambda \mathbin{/} \si{\micro\metre}$} & {$\theta_1$} & {$\theta_2$} & {$\theta$} & {$\frac{\theta}{d}$}\\
            \midrule
            1.060 & 4.349 & 4.200 & 0.075 & 54.862 \\
            1.290 & 4.364 & 4.260 & 0.052 & 38.179 \\ 
            1.450 & 4.373 & 4.275 & 0.049 & 35.826 \\
            1.720 & 4.362 & 4.255 & 0.053 & 39.249 \\ 
            1.960 & 4.390 & 4.191 & 0.100 & 73.471 \\ 
            2.156 & 4.374 & 4.254 & 0.060 & 44.061 \\ 
            2.340 & 3.347 & 3.220 & 0.063 & 46.521 \\ 
            2.510 & 3.586 & 3.365 & 0.110 & 81.171 \\ 
            2.560 & 3.023 & 2.967 & 0.028 & 20.426 \\ 
            \bottomrule            
        \end{tabular}
    \end{table}

    \begin{table}
        \centering
        \caption{Die gemessenen Winkel einer $\ce{InGaAs}$-Probe der Länge $d = \SI{1.296}{\milli\metre}$ und der Dotierung $N = \SI{2.8e18}{\per\centi\metre\tothe{3}}$. }
        \label{tab:probe_28}
        \begin{tabular}{S[table-format=1.3] S[table-format=1.3] S[table-format=1.3] S[table-format=2.3] S[table-format=3.3]}
            \toprule
            {$\lambda \mathbin{/} \si{\micro\metre}$} & {$\theta_1$} & {$\theta_2$} & {$\theta$} & {$\frac{\theta}{d}$}\\
            \midrule
            1.060 & 4.330 & 4.150 & 0.090 & 69.804 \\
            1.290 & 2.971 & 2.851 & 0.060 & 46.349 \\
            1.450 & 2.944 & 2.788 & 0.078 & 60.153 \\
            1.720 & 2.986 & 2.818 & 0.084 & 65.091 \\
            1.960 & 3.092 & 2.880 & 0.106 & 81.700 \\
            2.156 & 3.202 & 3.177 & 0.012 & 9.539 \\
            2.340 & 3.495 & 3.156 & 0.170 & 130.967 \\
            2.510 & 3.823 & 3.407 & 0.208 & 160.370 \\
            2.560 & 3.247 & 2.987 & 0.130 & 100.105 \\
            \bottomrule            
        \end{tabular}
    \end{table}

\subsection{Ermittlung der effektiven Masse}

    \noindent Zur Berechnung der effektiven Masse wird nun an den Ausdruck
    \begin{equation*}
        \increment \theta = \left(\frac{\theta}{d}\right)_\text{rein} - \left(\frac{\theta}{d}\right)_\text{Probe}
    \end{equation*}
    eine lineare Ausgleichsfunktion der Form 
    \begin{equation*}
        \increment \theta = a \cdot \lambda^2 + b
    \end{equation*}
    angepasst. Hier wird \textit{curve-fit} von Scipy genutzt.
    Anschließend kann aus der Steigung $a$ wie folgt die effektive Masse bestimmt werden
    \begin{equation}
        m^* = \sqrt{\frac{e^3}{8 \pi^2 \epsilon_0 c^3} \frac{N\cdot B_\text{max}}{n} \frac{1}{a}}\quad .
        \label{eqn:effmass}
    \end{equation}
    Dabei ist $N$ die Donatorenkonzentration, $n$ der Brechungsindex von reinem $\ce{GaAs}$, welcher sich zu $\num{3.3543}$ ergibt \cite{n}. 

    \noindent Zu der Probe mit $N = \SI{1.2e18}{\per\centi\metre\tothe{3}}$ findet sich in \autoref{fig:probe_12_diff} die Gerade grafisch dargestellt. 
    Die Parameter aus der Ausgleichsrechnung ergeben sich zu 
    \begin{align*}
        a &= \SI{-8(4)e12}{1\per\metre\tothe{3}} & \text{und} & & b &= \SI{88(17)}{1\per\metre}\quad .
    \end{align*}
    Nach dem Einsetzen aller Größen in SI-Einheiten in Gleichung \eqref{eqn:effmass}, ergibt sich 
    \begin{equation*}
        m^* = \SI{6.5(16)e-32}{\kilo\gram} = \left(\num{0.0712(180)}\right)\cdot m_\text{e}
    \end{equation*}
    als Wert für die effektive Masse. 

    \begin{figure}[H]
        \centering
        \includegraphics[width=0.8\textwidth]{build/plots/probe1_diff.pdf}
        \caption{Die Differenz der normierten mit $N = \SI{1.2e18}{\per\centi\metre\tothe{3}}$ dotierten $\ce{InGaAs}$-Probe und undotierten Probe gegen 
        die Wellenlänge zum Quadrat aufgetragen inklusive einer linearen Ausgleichsrechnung.}
        \label{fig:probe_12_diff}
    \end{figure}

    \noindent Die Ergebnisse der mit $N = \SI{2.8e18}{\per\centi\metre\tothe{3}}$ dotierten Probe sind grafisch in \autoref{fig:probe_28_diff} zu sehen. 
    Für die Ausgleichsrechnung ergeben sich die Parameter zu 
    \begin{align*}
        a &= \SI{4(8)e12}{1\per\metre\tothe{3}} & \text{und} & & b &= \SI{74(33)}{1\per\metre}\quad .
    \end{align*}
    Die effektive Masse der Elektronen berechnet sich zu 
    \begin{equation*}
        m^* = \SI{1.3(12)e-31}{\kilo\gram} = \left(\num{0.148(134)}\right)\cdot m_\text{e} \quad ,
    \end{equation*}
    nach Gleichung \eqref{eqn:effmass}.

    \begin{figure}[H]
        \centering
        \includegraphics[width=0.8\textwidth]{build/plots/probe2_diff.pdf}
        \caption{Die Differenz der normierten mit $N = \SI{2.8e18}{\per\centi\metre\tothe{3}}$ dotierten $\ce{InGaAs}$-Probe und dotierten Probe gegen 
        die Wellenlänge zum Quadrat aufgetragen, inklusive einer linearen Ausgleichsrechnung.}
        \label{fig:probe_28_diff}
    \end{figure}