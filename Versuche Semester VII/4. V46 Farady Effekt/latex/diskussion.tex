\newpage
\section{Diskussion}

\noindent Die in diesem Versuch ermittelten effektiven Massen sind in \autoref{tab:vgl} aufgelistet und mit ihrem Theoriewert verglichen. 
Die relative Abweichung berechnet sich dabei nach
\begin{equation*}
    \increment m^* = 1 - \frac{m^*_\text{lit}}{m^*_\text{exp}} \quad .
\end{equation*}

\begin{table}
    \centering
    \caption{Die im Versuch ermittelten Größen im Vergleich mit ihren Literaturwerten.}
    \label{tab:vgl}
    \begin{tabular}{c c c c}
        \toprule
        {$N \mathbin{/} \si{\centi\metre\tothe{-3}}$} & {$m^*_\text{exp} \mathbin{/} m_\text{e}$} & {$m^*_\text{lit} \mathbin{/} m_\text{e}$ \protect \cite{effmasse}} & {$\increment m^* \mathbin{/} \si{\percent}$}\\ 
        \midrule
        $\num{1.2e18}$ & $\num{0.0071(18)}$ & $\num{0.067}$ & 89,4 \\
        $\num{2.8e18}$ & $\num{0.0150(13)}$ & $\num{0.067}$ & 78,0 \\
        \bottomrule
    \end{tabular}
\end{table}

\noindent In \autoref{tab:vgl} ist zu erkennen, dass die ermittelten Werte stark von den Literaturwerten entfernt abweichen. Jedoch ist dies verständlich bei 
Betrachtung der linearen Ausgleichsrechnungen. Die Linearität ist nicht sehr deutlich zu sehen. Daher sind die Abweichungen als Folgefehler zu betrachten. \\
Zusätzlich kommen wahrscheinlich viele Abweichungen aus der Messmethode. Die Minimierung erfolgt nach Augenmaß am Oszilloskop und das Goniometer ist auch nur 
bedingt präzise abzulesen.

\noindent Der Versuchsablauf lief ohne größere Probleme. Die Spannung der Halogenlampe wurde früh während der Messungen auf $\SI{8}{\volt}$ gestellt, da sonst die 
Photowiderstände überladen waren. Einmal ist die Linse nach der Lichtquelle in seiner Halterung heruntergerutscht, dies wurde aber schnell bemerkt und die 
letzten 4 Messungen davor wiederholt.
