\newpage
\section{Diskussion}
Die Durchführung des Versuchs lief ohne Probleme ab. Die Justagemessungen haben erwarteten Ergebnisse geliefert. 
Der aus dem Rockingscan bestimmte Geometriewinkel $\alpha_\t{g,2} = \SI{0.82}{\degree} $ besitzt nur eine relative Abweichung von 
$\SI{2.223}{\percent}$ von dem errechneten Wert $\alpha_\t{g,1}= \SI{0.802}{\degree}$. 
Dies ist ein sehr gutes Ergebnis, welches zeigt, dass die Justagemessungen erfolgreich waren.\\
Die in Abbildung \ref{img:refl1} dargestellte theoretische Reflektivität einer glatten Siliziumoberfläche stimmt nur ungefähr mit den Messwerten überein, 
da sie eine etwas stärkere Krümmung besitzt. Ein Ansatz um dies zu erklären wäre, dass die Theoriekurve für eine glatte Oberfläche gilt, was für die reale Oberfläche nicht gilt.
Dies könnte zu der leicht unterschiedlichen Krümmung der Kurve führen.\\
Die manuelle Ausgleichsrechnung für den Parratt-Algorithmus hat viele Probleme hervorgebracht. Zum einen haben gängige Fit-Algorithmen keine passenden Ergebnisse geliefert.
Zum anderen war das manuelle Anpassen der Parameter schwierig, da das Variieren eines Parameters die Kurve sehr stark in vielen Eigenschaften verändern kann.
Das nacheinander Variieren war deswegen nicht möglich, wodurch das Fiten chaotisch war und es anspruchsvoll war ei zufriedenstellenes Ergebnis zu erreichen.
Die in Tabelle \ref{tab:rel} aufgetragenen Ergebnisse und relativen Abweichungen der Theoriewerte\cite{V44} unterstreichen dies auch. 
Die Dispersion von Polysterol zeigt eine starke Abweichung von dem Referenzwert. 
Dafür das nur ein ungenauer Fit durchgeführt wurde, ist die von Silizium aber überraschend nah an der Literatur und liefert ein gutes Ergebnis.

\begin{table}[ht]
    \centering
    \caption{Relative Abweichung von den Literaturwerten \protect\cite{V44} für die einzelnen Parameter des Parratt-Algorithmus.}
    \label{tab:rel}
        \begin{tabular}{S [table-format=4.4] S [table-format=4.4] S [table-format=4.4] S [table-format=4.4]}
        \toprule
            {Größe} & {Fit-Parameter} &{Literaturwert}& {$\text{relative Abweichung} $} \\
        \midrule
        $\delta_\t{Poly}$ & $\SI{1e-6}{}$      & $\SI{3.5e-6}{}$ & $\SI{71.43}{\percent}$\\
        $\delta_\t{Si}$   & $\SI{8.15e-6}{}$   & $\SI{7.6e-6}{}$ & $\SI{-7.23}{\percent}$\\
            \hline
        ${d}$             & $\SI{8.8e-8 }{\metre}$   & $\SI{8.68e-8}{}$ & $\SI{-1.40}{\percent}$\\
        \bottomrule
    \end{tabular}
    \label{tab:rel}
\end{table} 

\noindent
Die Schichtdicke $d$ des Parratt-Algorithmus wird mit der aus den Kiessig-Oszillationen berechneten verglichen. 
Die beiden besitzen auch nur eine sehr geringe Abweichung voneinander, obwohl der Parratt-Fit problematisch war\\
Alles in allem hat der Versuch gute Ergebnisse geliefert, vor allem in Anbetracht des händischen Parratt-Fits.
\newpage

